\documentclass[a4paper,11pt]{article}
\usepackage{apacite}
\usepackage{url}
\usepackage{graphicx}
\graphicspath{ C:\Users\javie\Desktop\M6 }
\providecommand{\keywords}[1]{\textbf{\textit{Palabras clave---}} #1}
\begin{document}
\title{Espermatogénesis in vitro}
\author{Javier Villegas Salmerón- Universidad de Granada}
\maketitle
\section{Resumen}
$https://github.com/Javizuma/proyecto_final$\\
La generación de espermatozoides funcionales in vitro ha sido un objetivo durante casi un siglo. Hasta hace poco, los investigadores solo han logrado reproducir las primeras etapas de la espermatogénesis, lo que no es sorprendente dado que es un proceso que requiere esfuerzos combinados de las células germinales y varias células somáticas.
El proceso de espermatogénesis, si bien ha sido estudiado satisfactoriamente en diversas especies, ha sido especialmente investigado en ratones, pues este es el modelo animal  más interesante de cara a obtener conocimiento también sobre el ser humano debido a las similitudes que presentan ambas especies. Respecto al ciclo de espermatogénesis en si, existen diferencias, sobre todo en la duración del ciclo y obviamente derivadas de que los ratones son usados como modelos  precisamente porque tienen un tiempo de vida bastante corto que facilita su manejo y experimentación. 
En el siguiente articulo se realizará una reunión del conocimiento generado hasta el momento en esta rama y nos situaremos en el contexto de los estudios que se realizan actualmente. 
\keywords{Espermatogénesis, in vitro, PGCs}
\section{Introducción}
En ratón, los progenitores de las células primordiales germinales (PGCs) derivan del epiblasto (capa más gruesa del disco embrionario) del blastocisto (fase embrionaria) en el saco vitelino, en respuesta a la proteína BMP (proteína morfogénica ósea). Alrededor del día 6 embrionario, poco antes de que el epiblasto se separe en tres capas: ectodermo, mesodermo y endodermo, las células pluripotenciales de la zona posterior proximal del epiblasto se diferencian en PGCs. Tras esto, comienzan a migrar y entre los días 7 y 8 ya se encuentran en el alantoides (membrana que rodea el embrión), entonces son transferidas al epitelio del intestino grueso,  al día 9 o 10 empiezan a migrar al mesenterio (estructura que une intestino y abdomen), alcanzándolo al día 10 u 11. Por tanto, desde el intestino grueso alcanzan la cresta gonadal entre los días 9,5 y 11 y ya se les denomina como PGC postmigratorias. Sobre el día 13.5 se diferencian en proespermatogonias y estas quedan retenidas dentro del compartimento luminal de los túbulos seminíferos hasta el nacimiento. 

\end{document}