\documentclass[a4paper,11pt]{article}
\usepackage{apacite}
\usepackage{url}
\usepackage{graphicx}
\graphicspath{ C:\Users\javie\Desktop\M6 }
\providecommand{\keywords}[1]{\textbf{\textit{Palabras clave---}} #1}
\begin{document}
\title{Espermatogénesis in vitro}
\author{Javier Villegas Salmerón- Universidad de Granada}
\maketitle
\section{Resumen}
$https://github.com/Javizuma/proyecto_final$\\
La generación de espermatozoides funcionales in vitro ha sido un objetivo durante casi un siglo. Hasta hace poco, los investigadores solo han logrado reproducir las primeras etapas de la espermatogénesis, lo que no es sorprendente dado que es un proceso que requiere esfuerzos combinados de las células germinales y varias células somáticas.
El proceso de espermatogénesis, si bien ha sido estudiado satisfactoriamente en diversas especies, ha sido especialmente investigado en ratones, pues este es el modelo animal  más interesante de cara a obtener conocimiento también sobre el ser humano debido a las similitudes que presentan ambas especies. Respecto al ciclo de espermatogénesis en si, existen diferencias, sobre todo en la duración del ciclo y obviamente derivadas de que los ratones son usados como modelos  precisamente porque tienen un tiempo de vida bastante corto que facilita su manejo y experimentación. 
En el siguiente articulo se realizará una reunión del conocimiento generado hasta el momento en esta rama y nos situaremos en el contexto de los estudios que se realizan actualmente. 

\end{document}